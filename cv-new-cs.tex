%-------------------------
% Životopis v LaTeXu
% Autor: Jake Gutierrez
% Založeno na: https://github.com/sb2nov/resume
% Licence: MIT
%------------------------

\documentclass[a4paper,11pt]{article}

\usepackage{latexsym}
\usepackage[empty]{fullpage}
\usepackage{titlesec}
\usepackage{marvosym}
\usepackage[usenames,dvipsnames]{color}
\usepackage{verbatim}
\usepackage{enumitem}
\usepackage[hidelinks]{hyperref}
\usepackage{fancyhdr}
\usepackage[czech]{babel} % Změna na češtinu
\usepackage{tabularx}
\input{glyphtounicode}


%----------MOŽNOSTI PÍSMA----------
% bezpatkové
% \usepackage[sfdefault]{FiraSans}
% \usepackage[sfdefault]{roboto}
% \usepackage[sfdefault]{noto-sans}
% \usepackage[default]{sourcesanspro}

% patkové
% \usepackage{CormorantGaramond}
% \usepackage{charter}


\pagestyle{fancy}
\fancyhf{} % vymazat všechny hlavičkové a patičkové pole
\fancyfoot{}
\renewcommand{\headrulewidth}{0pt}
\renewcommand{\footrulewidth}{0pt}

% Nastavení okrajů
\addtolength{\oddsidemargin}{-0.5in}
\addtolength{\evensidemargin}{-0.5in}
\addtolength{\textwidth}{1in}
\addtolength{\topmargin}{-0.5in}
\addtolength{\textheight}{1.0in}

\urlstyle{same}

\raggedbottom
\raggedright
\setlength{\tabcolsep}{0in}

% Formátování sekcí
\titleformat{\section}{
  \vspace{-4pt}\scshape\raggedright\large
}{}{0em}{}[\color{black}\titlerule \vspace{-5pt}]

% Zajištění, aby generovaný PDF byl strojově čitelný/parsovatelný ATS
\pdfgentounicode=1

%-------------------------
% Vlastní příkazy
\newcommand{\resumeItem}[1]{
  \item\small{
    {#1 \vspace{-2pt}}
  }
}

\newcommand{\resumeSubheading}[4]{
  \vspace{-2pt}\item
    \begin{tabular*}{0.97\textwidth}[t]{l@{\extracolsep{\fill}}r}
      \textbf{#1} & #2 \\
      \textit{\small#3} & \textit{\small #4} \\
    \end{tabular*}\vspace{-7pt}
}

\newcommand{\resumeSubSubheading}[2]{
    \item
    \begin{tabular*}{0.97\textwidth}{l@{\extracolsep{\fill}}r}
      \textit{\small#1} & \textit{\small #2} \\
    \end{tabular*}\vspace{-7pt}
}

\newcommand{\resumeProjectHeading}[2]{
    \item
    \begin{tabular*}{0.97\textwidth}{l@{\extracolsep{\fill}}r}
      \small#1 & #2 \\
    \end{tabular*}\vspace{-7pt}
}

\newcommand{\resumeSubItem}[1]{\resumeItem{#1}\vspace{-4pt}}

\renewcommand\labelitemii{$\vcenter{\hbox{\tiny$\bullet$}}$}

\newcommand{\resumeSubHeadingListStart}{\begin{itemize}[leftmargin=0.15in, label={}]}
\newcommand{\resumeSubHeadingListEnd}{\end{itemize}}
\newcommand{\resumeItemListStart}{\begin{itemize}}
\newcommand{\resumeItemListEnd}{\end{itemize}\vspace{-5pt}}

%-------------------------------------------
%%%%%%  ŽIVOTOPIS ZAČÍNÁ ZDE  %%%%%%%%%%%%%%%%%%%%%%%%%%%%


\begin{document}

%----------HLAVIČKA----------
% \begin{tabular*}{\textwidth}{l@{\extracolsep{\fill}}r}
%   \textbf{\href{http://sourabhbajaj.com/}{\Large Sourabh Bajaj}} & Email : \href{mailto:sourabh@sourabhbajaj.com}{sourabh@sourabhbajaj.com}\\
%   \href{http://sourabhbajaj.com/}{http://www.sourabhbajaj.com} & Mobile : +1-123-456-7890 \\
% \end{tabular*}

\begin{center}
    \textbf{\Huge \scshape Tomáš Moravec} \\ \vspace{1pt}
    \small +420 736 768 468 $|$ \href{mailto:info@tomas-moravec.dev}{\underline{info@tomas-moravec.dev}} $|$ 
    \href{https://www.linkedin.com/in/moravectomas}{\underline{linkedin.com/in/moravectomas}} $|$
    \href{https://github.com/tomasmo-dev}{\underline{github.com/tomasmo-dev}}
\end{center}


%-----------VZDĚLÁNÍ-----------
\section{Vzdělání}
  \resumeSubHeadingListStart
    \resumeSubheading
      {Střední škola a Vyšší odborná škola aplikované kybernetiky s.r.o.}{Hradec Králové, Česká republika}
      {Programování \& IT sítě}{září 2021 -- současnost}%červen 2025}
  \resumeSubHeadingListEnd


%-----------ZKUŠENOSTI-----------
\section{Zkušenosti}
  \resumeSubHeadingListStart

    \resumeSubheading
      {Softwarový programátor}{červen 2023 -- současnost}
      {}{Vzdáleně}
      \resumeItemListStart
        \resumeItem{Vyvinul jsem účetní software pro vytváření faktur pro studenty}
        \resumeItem{Vyvinul jsem aplikaci na využívání OpenAi API k efektivní opravě testových otázek}
        \resumeItem{Vytvořil jsem nástroj na export testových otázek z excelových souborů do databáze MySql}
      \resumeItemListEnd
      
% -----------Hlavička pro více pozic-----------
%    \resumeSubSubheading
%     {Software Engineer I}{říjen 2014 - září 2016}
%     \resumeItemListStart
%        \resumeItem{Apache Beam}
%          {Apache Beam je unifikovaný model pro definici paralelních zpracovatelských potrubí pro zpracování dat ve formě bloků a ve formě datově paralelních toků}
%     \resumeItemListEnd
%    \resumeSubHeadingListEnd
%-------------------------------------------

  \resumeSubHeadingListEnd


%-----------PROJEKTY-----------
\section{Projekty}
    \resumeSubHeadingListStart
      \resumeProjectHeading
          {\textbf{Vlastní webový server v C\#} $|$ \emph{C\#, CSS, JS, HTML}}{ 24. února 2022 -- 21. února 2023 }
          \resumeItemListStart
            \resumeItem{Vyvinul jsem program na hostování v C\#, který používá trasování definované v JSON}
            \resumeItem{Obsahuje vestavěné vlastní systém na zaznamenávání informaci/errorů a podporuje HTML, CSS, JS a obrázky}
          \resumeItemListEnd
      \resumeProjectHeading
          {\textbf{Jednoduchá online bankovní aplikace} $|$ \emph{C\#, ASP.NET, MSSQL,	CSHTML, CSS, JS \hspace{0.5em}}}{listopad 2023 -- leden 2024}
          \resumeItemListStart
            \resumeItem{Vytvořil jsem aplikaci ASP.NET Core MVC pro simulaci online bankovnictví}
            \resumeItem{Aplikace používá MSSQL databázi pro trvalé uložení dat}
            \resumeItem{Aplikace umožňuje registraci uživatelů a převody peněz}
            \resumeItem{Obsahuje také administrátorskou stránku, která zobrazuje všechny uživatele a jejich transakce s veškerými osobními údaji}
          \resumeItemListEnd
    \resumeSubHeadingListEnd



%
%-----------DOVEDNOSTI-----------
\section{Technické dovednosti}
 \begin{itemize}[leftmargin=0.15in, label={}]
    \small{\item{
     \textbf{Jazyky}{: C\#, C/C++ (Arduino/ESP32), SQL (Postgres, MSSQL, MySql), JavaScript, HTML/CSS/CSHTML, PHP} \\
     \textbf{Frameworky}{: ASP.NET Core MVC, Bootstrap, Picocss} \\
     \textbf{Vývojářské nástroje}{: Git, Docker, Docker Compose, Google Cloud Platform, VS Code, Visual Studio, PhpStorm, Rider} \\
     \textbf{Knihovny}{: EntityFramework, HtmlAgilityPack}
    }}
 \end{itemize}


%-------------------------------------------
\end{document}
